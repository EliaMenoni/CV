%----------------------------------------------------------------------------------------
%	PACKAGES AND OTHER DOCUMENT CONFIGURATIONS
%----------------------------------------------------------------------------------------

\documentclass[9pt]{developercv} % Default font size, values from 8-12pt are recommended

%----------------------------------------------------------------------------------------

\begin{document}

%----------------------------------------------------------------------------------------
%	TITLE AND CONTACT INFORMATION
%----------------------------------------------------------------------------------------

\begin{minipage}[t]{0.45\textwidth} % 45% of the page width for name
	\vspace{-\baselineskip} % Required for vertically aligning minipages

	% If your name is very short, use just one of the lines below
	% If your name is very long, reduce the font size or make the minipage wider and reduce the others proportionately
	\colorbox{green_E}{{\HUGE\textcolor{white}{\textbf{\MakeUppercase{Elia}}}}} % First name

	\colorbox{green_E}{{\HUGE\textcolor{white}{\textbf{\MakeUppercase{Menoni}}}}} % Last name

	\vspace{6pt}

	{\huge Studente} % Career or current job title
\end{minipage}
\begin{minipage}[t]{0.275\textwidth} % 27.5% of the page width for the first row of icons
	\vspace{-\baselineskip} % Required for vertically aligning minipages

	% The first parameter is the FontAwesome icon name, the second is the box size and the third is the text
	% Other icons can be found by referring to fontawesome.pdf (supplied with the template) and using the word after \fa in the command for the icon you want
	\icon{MapMarker}{12}{Aulla, Toscana, Italia}\\
	\icon{Phone}{12}{+39 392 539 8587}\\
\end{minipage}
\begin{minipage}[t]{0.275\textwidth} % 27.5% of the page width for the second row of icons
	\vspace{-\baselineskip} % Required for vertically aligning minipages
  % menonielia@pec.net
  \icon{At}{12}{\href{mailto:menonielia00@gmail.com}{menonielia00@gmail.com}}\\
  \icon{At}{12}{\href{mailto:menonielia@pec.net}{menonielia@pec.net}}\\
	% The first parameter is the FontAwesome icon name, the second is the box size and the third is the text
	% Other icons can be found by referring to fontawesome.pdf (supplied with the template) and using the word after \fa in the command for the icon you want
	% \icon{Globe}{12}{\href{https://about.me/eliamenoni}{about.me/eliamenoni}}\\
	\icon{Github}{12}{\href{https://github.com/EliaMenoni}{github/EliaMenoni}}\\
	%\icon{Twitter}{12}{\href{https://twitter.com/@alyxvance}{@alyxvance}}\\
\end{minipage}

\vspace{0.5cm}

%----------------------------------------------------------------------------------------
%	INTRODUCTION, SKILLS AND TECHNOLOGIES
%----------------------------------------------------------------------------------------

\cvsect{Chi sono}

\begin{minipage}[t]{0.4\textwidth} % 40% of the page width for the introduction text
	\vspace{-\baselineskip} % Required for vertically aligning minipages

	Sono uno studente del corso di laurea triennale in IT presso l'Università degli studi di Pisa. La mia passione per il mondo dell'informatica
	comincia in tenera età e si rafforza durante la formazione scolastica fino ad oggi.	Ho grandi ambizioni sia negli studi che nella vita: punto a laurearmi in informatica con buoni voti e a conseguire la laurea magistrale incentrando i miei studi
	sulla sicurezza informatica.\\
\end{minipage}
\hfill % Whitespace between
\begin{minipage}[t]{0.5\textwidth} % 50% of the page for the skills bar chart
	\vspace{-\baselineskip} % Required for vertically aligning minipages
	\cvskill{C/C++}{10}
	\cvskill{PHP}{7}
	\cvskill{Python}{8}
	\cvskill{Java}{7}
	\cvskill{Git}{8}
\end{minipage}

%\begin{center}
%	\bubbles{5/Eclipse, 6/git, 4/Office, 3/Inkscape, 3/Blender}
%\end{center}

%----------------------------------------------------------------------------------------
%	EXPERIENCE
%----------------------------------------------------------------------------------------

\cvsect{Esperienze}

\begin{entrylist}
	\entry
	{2017}
	{Stage - Programmatore}
	{Zucchetti SPA}
	{Stage, scolastico, della durata di 10gg come programmatore presso la software house Zucchetti SPA\\ \texttt{C}\slashsep\texttt{C++}\slashsep\texttt{Visual Fox}}
\end{entrylist}

%----------------------------------------------------------------------------------------
%	EDUCATION
%----------------------------------------------------------------------------------------

\cvsect{Studi e Certificati}

\begin{entrylist}
	\entry
	{2019 -- Oggi}
	{Laurea triennale in informatica}
	{Università degli studi di Pisa (UNIPI)}
	{Il corso di laurea comprende complementi di matematica (analisi, algebra lineare, ecc.) e fisica, uniti a materie specifiche di progettazione, analisi e implementazione
		di progetti inerenti al mondo dell'informatica.\\
		\texttt{C}\slashsep\texttt{C++}}
	\entry
	{2014 -- 2019}
	{Diploma}
	{Fossati da Passano, La Spezia}
	{Ho conseguito il diploma e la qualifica di "Perito informatico e delle telecomunicazioni" presso l'istituto tecnico Fossati da Passano di La Spezia con una votazione di 90/100.
		Durante gli anni di studio ho appreso conoscenze approfondite che spaziano dallo svilippo SW a nozioni teoriche e pratiche sui sistemi informatici.\\
		\texttt{C}\slashsep\texttt{C++}\slashsep\texttt{Java}\slashsep\texttt{PHP}\slashsep\texttt{SQL}\slashsep\texttt{HTML \& CSS}\slashsep\texttt{JS}}
	\entry
	{2017 -- 2018}
	{Partecipazione CyberChallenge.IT}
	{Università degli studi di Genova (UNIGE)}
	{Durante il quarto anno di superiori sono entrato a far parte della squadra ligure, per il concorso CyberChallenge.IT organizzato dal CINI, insieme ad altri 19 studenti selezionati su oltre
		600 candidati in tutta la Liguria. In tre mesi siamo stati formati sulle varie branche della sicurezza informatica che spaziano dalla sicurezza software alla sicurezza web.\\
		\texttt{C}\slashsep\texttt{C++}\slashsep\texttt{Python}\slashsep\texttt{PHP}\slashsep\texttt{SQL}\slashsep\texttt{HTML}}
	\entry
	{2018}
	{ECDL Base, Standard, Full Standard, IT Security}
	{AICA}
	{Certificato attestante la competenza nell'uso del computer.\\
		\texttt{Office}\slashsep\texttt{G Suite}}
	\entry
	{2017}
	{Certificato per la lingua inglese PET}
	{International House}
	{Certificato attestante la conoscenza della lingua inglese emesso dalla International House.}
\end{entrylist}

%----------------------------------------------------------------------------------------
%	ADDITIONAL INFORMATION
%----------------------------------------------------------------------------------------

\begin{minipage}[t]{0.3\textwidth}
	\vspace{-\baselineskip} % Required for vertically aligning minipages

	\cvsect{Lingue}

	\textbf{Italiano} - Madre lingua\\
	\textbf{Inglese} - Esaustivo (PET)\\
\end{minipage}
\hfill
\begin{minipage}[t]{0.3\textwidth}
	\vspace{-\baselineskip} % Required for vertically aligning minipages

	\cvsect{Hobbies}

	Mi piace leggere libri inerenti alla sicurezza informatica: dall'analisi dei BigData ai grandi eventi come il libro "Errore di sistema" di Edward Snowden.
\end{minipage}
\hfill
\begin{minipage}[t]{0.3\textwidth}
	\vspace{-\baselineskip} % Required for vertically aligning minipages

	\cvsect{Interessi}

	Oltre la materia che studio, la mi più grande passione, mi piace la musica. Suono o ho suonato il pianoforte, il sax, l'armonica e la chitarra.
\end{minipage}

%----------------------------------------------------------------------------------------

\end{document}
